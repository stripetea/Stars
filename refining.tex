Теперь мы очистим список от источников, который могли попасть туда случайно. Это могут быть галактики, например, или горячие звёзды, потому что у них тоже присутствует избыток ультрафиолета.

\section{Удаление источников известного типа}
В первую очередь нужно очистить список от источников, которые заведомо не являются звёздами типа Т Тельца. Это могут быть галактики или звёзды, отношение которых к какому-то иному типу уже установлено.

С помощью сервиса Simbad можно узнать, чем является интересующий нас источник, имея лишь его координаты. Simbad (Set of Identifications, Measurements, and Bibliography for Astronomical Data ) -- это база данных об астрономических объектах, лежащих вне Солнечной системы. Таким образом, загрузив в неё список координат возможных кандидатов, мы можем узнать их тип, если он когда-либо и кем-либо был определён.

Из 302 источников нашего списка Simbad идентифицировал лишь 71, причём 56 из них являются галактиками, и 15 -- звёздами. Галактики должны быть выброшены из списка. Все звёзды не отнесены ни к какому типу, поэтому их следует оставить.  

Остальные источники не идентифицированы вовсе, то есть могут являться любым объектом. Они тоже, разумеется, должны быть оставлены в списке.

\section{Поиск галактик по собственным движениям}
Многие неидентифицированные Simbad источники могут также быть галактиками. Чтобы избавиться от них, мы можем сравнить их собственные движения. Те объекты, чьи собственные движения очень малы, то есть лежат в пределах погрешности наблюдений, с высокой вероятности лежат вне нашей Галактики.

Чтобы определить собственные движения, нужно заснять интересующую нас область и сравнить положения объектов с положениями, зафиксированными более ранними наблюдениями. Это может быть каталог DSS, который составлен на основе наблюдений, проведённых более 50 лет назад.

К сожалению, нам не удалось найти свежих фотографий неба в нужной области. 

В каталоге UCAC4 содержится информация о собственных движениях, но в него входят только объекты, заведомо являющиеся звёздами, поэтому он не может нам помооочь.

\section{Оценка эффективных температур}
Некоторые попавшие в список источники могут иметь ультрафиолетовый избыток просто потому что у них очень большая температура. Оцениваем её в VOSA, отбрасываем все источники горячее 7000 К.

