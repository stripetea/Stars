Теперь мы очистим список от источников, который могли попасть туда случайно. Это могут быть галактики, например, или горячие звёзды, потому что у них тоже присутствует избыток ультрафиолета.

\section{Удаление источников известного типа}
В первую очередь нужно очистить список от источников, которые заведомо не являются звёздами типа Т Тельца. Это могут быть галактики или звёзды, отношение которых к какому-то иному типу уже установлено.

С помощью сервиса Simbad можно узнать, чем является интересующий нас источник, имея лишь его координаты. Simbad (Set of Identifications, Measurements, and Bibliography for Astronomical Data ) -- это база данных об астрономических объектах, лежащих вне Солнечной системы. Таким образом, загрузив в неё список координат возможных кандидатов, мы можем узнать их тип, если он когда-либо и кем-либо был определён.

Из 302 источников нашего списка Simbad идентифицировал лишь 74, причём 59 из них являются галактиками, и 15 -- звёздами. Галактики должны быть выброшены из списка. Почти все звёзды не отнесены ни к какому спектральному классу, поэтому их следует оставить. Две, имеющие класс G5, нам не подходят. Ещё две относятся к классу K2, одна к K0 и является двойной. K может и нужны

Остальные источники не идентифицированы вовсе, то есть могут являться любым объектом. Они тоже, разумеется, должны быть оставлены в списке.

\section{Поиск галактик по собственным движениям}
Многие неидентифицированные Simbad источники могут также быть галактиками. Чтобы избавиться от них, мы можем сравнить их собственные движения. Те объекты, чьи собственные движения очень малы, то есть лежат в пределах погрешности наблюдений, с высокой вероятности лежат вне нашей Галактики.

Чтобы определить собственные движения, нужно заснять интересующую нас область и сравнить положения объектов с положениями, зафиксированными более ранними наблюдениями. Это может быть каталог DSS, который составлен на основе наблюдений, проведённых более 50 лет назад.

К сожалению, нам не удалось найти свежих фотографий неба в нужной области. 

В каталоге UCAC4 содержится информация о собственных движениях, но в него входят только объекты, заведомо являющиеся звёздами, поэтому он не может нам помооочь.

Как ни странно, три объекта, которые в Simbad описаны как галактики, оказались в каталоге UCAC4. Возможно, это связано с довольно большим радиусом кросс-корреляции, обусловленным точностью GALEX. То есть в круг радиусом 3 угловых секунды попадает и звезда, и галактика. Значит, и у других источников такие совпадения могли случиться. Эти объекты должны быть также отброшены, потому что GALEX, неспособный разрешить их, зафиксировал избыток суммарно галактики и звезды.

\section{Оценка эффективных температур}
Некоторые попавшие в список источники могут иметь ультрафиолетовый избыток просто потому что у них очень большая температура. 

Для оценки эффективной температуры использовалась виртуальная обсерватория VOSA (Virtual Observatory SED Analyzer). Она позволяет построить распределение спектральной энергии (spectral energy distribution, SED) для источника, звёздные величины которого загружены пользователем, а также найти другую фотометрическую информацию об источнике с указанными координатами.

С помощью SED сервис может оценить чернотельную температуру источников, зная также расстояние до него и величину поглощения $A_v$. Расстояние до области нам известно, оно равно 415 пк [ссылка]. Поглощение, по оценкам, данным в работах [А И] и [другая], не превышает 0.3-0.5. Для оценки бралась величина $A_v=0.3$.

Согласно полученным оценкам, температуры кандидатов лежат в диапазоне от 3000 K до 9000 K. Отбросив все источники горячее 7000 K, сократим список ещё на 40 строк. После этого этапа в списке остаётся 206 кандидатов.


