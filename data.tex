Данные мы брали из интернета, но это ничего страшного, они же в открытом дотупе лежат, и так и надо, зря что ли люди делали.

\section{GALEX}
Космическая миссия. О телескопе, звёздных величинах, фильтрах. Как выглядит спектр Т Таури в этих фильтрах.

О точности и покрытии неба. Картинка насчёт того как он покрывает мне область.

Первоначальный список источников из квадрата. Сколько. Какие данные

\section{2MASS и UCAC4}
Зачем нам вообще нужны эти данные. 2MASS для построения цветовых диаграмм, UCAC4 для оценки эффективных температур.

О каталогах, фильтрах, точности. В GALEX должно быть много ложных источников из-за их слабости и детектирования на пределе возможностей телескопа. Они пропадут при матчинге с нормальными каталогами
 
Кросс-идентификация первого списка с каталогами 2MASS и UCAC4. Радиус поиска. Какие откидывать? Новый список.

\section{Используемые инструменты}
casjobs для получения данных с галекса. visier для кросс-матчинга.
