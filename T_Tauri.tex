
\section{Звёзды типа Т Тельца}
Звёзды типа Т Тельца - это маломассивные молодые звёзды, находящиеся на пути к главной последовательности. Обычно они находятся недалеко от отражательных или тёмных туманностей, оставшихся от газопылевого облака, из которого эти звёзды сформировались. Эти звёзды находятся в той части диаграммы Герцшпрунга - Рассела, которая соответствует звёздам с массами около 2-3 солнечных. С точки зрения звёздной эволюции они находятся в стадии гравитационного сжатия и, как молодые объекты, имеют близкую к солнечной металличность \cite{dod2013}. Характерными чертами также является избыточная эмиссия в ИК и УФ дапазонах.

Выделяют два подтипа этих звёзд: классические звёзды типа Т Тельца (classical T Tauri stars, CTTS) и звёзды типа Т Тельца со слабыми линиями (weak-lined T Tauri stars). Звёзды обоих подтипов находятся на одной стадии эволюции, имеют малую массу, и их металличность близка к солнечной. Различие состоит в том, что в спектрах классических звёзд типа Т Тельца присутствуют сильные эмиссионные линии, указывающие на то, что эти звёзды проявляют определённого рода активность. У звёзд типа Т Тельца со слабыми линиями эмиссионные линии гораздо слабее. Граница между подтипами проводится по эквивалентной ширине линии H$\alpha$. 

Физическое различие между подтипами состоит в том, что у CTTS есть аккреционный диск, а у WTTS его нет \cite{petrov2003t}. Именно наличие аккреционного диска приводит к появлению ультрафиолетового избытка: вещество, падая на звезду, разогревается до высоких температур. У WTTS вокруг звезды также может находиться диск, но он пассивен и взаимодействует со звездой только посредством излучения.

%%%%%%%%%%%%%%%%%%%%%%%%%%%%%%%%%%%%%%%%%%%%%%%%%%%%%%%%%%%%%%%%%%%%%%%%%%%%%%%%%%%%%%%%%

\section{Спектральные особенности}
Звёзды типа Т Тельца относятся к классу переменных звёзд. Первоначально они были выделены в отдельный класс на основе чисто спектроскопических характеристик: наличия эмиссии в линиях H$\alpha$ и Fe I, а также класс светимости IV-V. 

Сейчас выделяются следующие критерии принадлежности к типу \cite{dod2013}:
\begin{itemize}
	\item Наличие поблизости тёмной или отражательной туманности;
	\item	Спектральный класс F5-M, класс светимости IV-V;
	\item	Эмиссия в линиях H и He I, а также нейтральных и однократно ионизированных металлов;
	\item	Сильная линия поглощения Li I 6707 A;
\end{itemize}

Присутствие линии Li указывает на молодость звёзд, так как согласно теоретическим расчётам литий быстро выгорает.

Эмиссионный спектр CTTS напоминает спектр солнечной хромосферы. Поэтому изначально считалось, что для них характерна высокая хромосферная активность \cite{krav2004}. Но ожидаемое в этой модели сильное рентгеновское излучение не нашло экспериментального подтверждения.

В настоящее время считается, что спектральные особенности CTTS обусловлены наличием аккреционного диска. Если также предположить наличие магнитного поля, направление которого не совпадает с осью вращения звезды, то удаётся объяснить асимметричность эмиссионных линий. Предполагается, что они образуются на границе магнитосферы. Также аккреция вещества на звезду вызывает возникновение джетов. Это биполярные узконаправленные струи газа, истекающие со звезды. Они наблюдаются обычно в запрещённых линиях [SiI], [OI].

Наличие протопланетных дисков и магнитного поля у многих звёзд типа Т Тельца подтверждается наблюдениями.

Чтобы отличить звёзды типа Т Тельца от других, нам нужно знать характеристики их спектров, в особенности те из них, которые можно наблюдать в фотометриях. Как следствие существования аккреционного диска и истечения вещества на звезду, у TTS наблюдается избыток излучения в различных спектральных диапазонах, а именно:
\begin{itemize}
	\item Избыток в инфракрасном диапазоне вплоть до миллиметровых длин волн, обусловленный как собственным излучением нагретого диска, так и переизлучением поглощённого им излучения звезды и джетов.
	\item Избыток в оптическом диапазоне – свечение плазмы, нагретой до температуры 7000-10000 K. Это так называемое вуалирование -- континуальное излучение нефотосферной природы. 
	\item Избыток в ультрафиолетовом диапазоне -- свечение плазмы с температурой электронов от 10000 K до 50000 K, причём присутствует как излучение в континууме, так и различные эмиссионные линии: нейтральные атомы (H I, O I, C I), однократно (C II, Si II, Fe II, Mg II, O I) и многократно (C IV, N V, O VI) ионизованные атомы, молекулярный водород.
	\item Избыток в рентгеновском диапазоне, вызванный высокой активностью магнитосферы звезды, свечением короны и ударными волнами в аккреционном диске.
\end{itemize}

%кортинка спектра T Tauri

%%%%%%%%%%%%%%%%%%%%%%%%%%%%%%%%%%%%%%%%%%%%%%%%%%%%%%%%%%%%%%%%%%%%%%%%%%%

\section{Методы поиска}
 Изначально главными критериями поиска были лишь самые основные характеристики звёзд типа Т Тельца, как-то: близость к молекулярным облакам, избыток в инфракрасном диапазоне, присутствие магнитного поля. Позже, с появлением широкомасштабных обзоров неба, стали учитываться эквивалентная ширина линии H $\alpha$, собственные движения звёзд и избыток излучения в рентгеновском диапазоне.

Также могут быть использованы оптические и инфракрасные данные и распределение спектральной энергии (SED) в этих диапазонах, в которых можно выделить черты, характерные для аккреционных дисков. Однако этими методами труднее обнаружить WTTS. Единственным действительно надёжным критерием является присутствие линии Li, как показатель молодости звезды \cite{dod2013}.

Несмотря на то, что у звёзд типа Т Тельца присутствует существенный ультрафиолетовый избыток, большинство исследований, направленных на их поиск, проводилось в инфракрасной и рентгеновской областях спектра. 

К УФ диапазону относятся многие характерные особенности T Tauri звёзд, поэтому в нём можно эффективно их изучать. С появлением обзора неба от миссии GALEX есть возможность использования и УФ области. Спектральные наблюдения в УФ осуществляются космическим телескопом Хаббл.



