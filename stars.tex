\documentclass[a4paper,12pt]{report}
\usepackage[T2A]{fontenc}
\usepackage[utf8]{inputenc}
%\usepackage[14pt]{extsizes}
\usepackage[english,russian]{babel}
\usepackage {titlesec, textcase} 
\usepackage{amssymb,amsfonts,amsmath,cite,enumerate,float}
\usepackage[dvips]{graphicx}
\usepackage {indentfirst}
\usepackage{pscyr}
\usepackage{cite,enumerate}

% поддержка гиперссылок; гиперссылки в pdf, должен быть последним загруженным пакетом
\ifx\pdfoutput\undefined
    \usepackage[unicode,dvips]{hyperref}
\else
    \usepackage[pdftex,colorlinks,unicode,bookmarks]{hyperref}
\fi

%\makeatletter
%\titleformat{\chapter}{\huge\bfseries}{\thechapter. }{0pt}{\huge} % Правим названия глав
%\bibliographystyle{utf8gost71u} % Оформляем библиографию по ГОСТ 7.1 (ГОСТ Р 7.0.11-2011, 5.6.7)
%\renewcommand{\@biblabel}[1]{#1.} % Заменяем библиографию с квадратных скобок на точку
%\makeatother

\makeatletter
%\titleformat{\chapter}{\huge\bfseries}{\thechapter. }{0pt}{\huge} % Правим названия глав
\bibliographystyle{IEEEtran} % Оформляем библиографию по ГОСТ 7.1 (ГОСТ Р 7.0.11-2011, 5.6.7)
\renewcommand{\@biblabel}[1]{#1.} % Заменяем библиографию с квадратных скобок на точку
\makeatother

\linespread{1.3} % Полуторный интвервал (ГОСТ Р 7.0.11-2011, 5.3.6)
\sloppy % Избавляемся от переполнений
\clubpenalty=10000 % Запрещаем разрыв страницы после первой строки абзаца
\widowpenalty=10000 % Запрещаем разрыв страницы после последней строки абзаца
\makeatletter
\renewcommand{\@biblabel}[1]{#1.} % Заменяем библиографию с квадратных скобок на точку:
\makeatother

% Меняем поля страницы
\usepackage{geometry}
\geometry{left=3cm}
\geometry{right=1.5cm}
\geometry{top=2cm}
\geometry{bottom=2cm}

% Меняем везде перечисления на цифра.цифра
\renewcommand{\theenumi}{\arabic{enumi}}
\renewcommand{\labelenumi}{\arabic{enumi}}
\renewcommand{\theenumii}{.\arabic{enumii}}
\renewcommand{\labelenumii}{\arabic{enumi}.\arabic{enumii}.}
\renewcommand{\theenumiii}{.\arabic{enumiii}}
\renewcommand{\labelenumiii}{\arabic{enumi}.\arabic{enumii}.\arabic{enumiii}.}
\renewcommand{\contentsname}{Содержание}
\renewcommand{\bibname}{Список использованных источников}


\newcommand{\angstrom}{\text{\normalfont\AA}}

\makeatletter
\titleformat{\chapter}{\huge\bfseries}{\thechapter. }{0pt}{\huge}
\makeatother

\addto\captionsrussian{
	\renewcommand{\contentsname}{Оглавление} % (ГОСТ Р 7.0.11-2011, 4)
	\renewcommand{\appendixname}{Приложение} % (ГОСТ Р 7.0.11-2011, 5.7)
	\renewcommand{\bibname}{Список литературы} % (ГОСТ Р 7.0.11-2011, 4)
}


\graphicspath{{images/}}

\begin{document}

\begin{titlepage}
\newpage
\begin{center}
Министерство образования и науки Российской Федерации\\
\vspace{1em} 
Федеральное государственное автономное образовательное учреждение\\
высшего профессионального образования\\
<<Московский физико-технический институт\\
(государственный университет)>>\\
\vspace{1em}
Факультет проблем физики и энергетики\\
\vspace{1em}
Кафедра нелинейных и динамических процессов в астрофизике и геофизике\\
\vspace{5em}
\textbf{\large\MakeTextUppercase{Подготовка программы наблюдений космической миссии "Спектр-УФ": отбор кандидатов в звёзды типа Т Тельца в созвездии Змеи}}\\
\vspace{1em}
Выпускная квалификационная работа\\
(бакалаврская работа)\\
\vspace{1em}
Направление подготовки 03.03.01 Прикладные математика и физика\\
\end{center}
\begin{flushleft}
\vspace{3em}
Выполнила:\\
студентка 183 группы \hrulefill Молярова Тамара Сергеевна\\
\vspace{3em}
Научный руководитель:\\
д.ф.-м.н., ведущий научный сотрудник \hrulefill Михаил Евгеньевич Сачков\\
\vspace{\fill}
\end{flushleft}
\begin{center}
Москва 2015
\end{center}
\end{titlepage}

\tableofcontents

\chapter{Введение}


\section{Цель работы}
Целью данной работы является поиск звёзд, относящихся к определённому классу: звёзд типа Т Тельца (T Tauri звёзд). 
Это предшественники звёзд, подобных Солнцу, а также планетарных систем. Поэтому их изучение очень важно для понимания процесса формирования Солнечной системы, её эволюции и образования планет. 

Как молодые звёзды, T Tauri обнаруживаются в областях звездообразования. Например, Т Тельца, по имени которой назван этот класс звёзд, расположена в молекулярном облаке Тельца (Taurus molecular cloud, TMC), где также находятся многие из известных T Tauri звёзд.

\section{Изучаемая область}

В данной работе изучается тёмная туманность, находящаяся в созвездиях Змея и Орёл (Serpens-Aquila Rift). Межзвёздная среда в ней находится в холодной фазе, то есть состоит из плотных и холодных облаков газа, в основном молекулярного водорода H$_{2}$. Именно из такого вещества формируются звёзды. Существуют исследования, подтверждающие, что в этой туманности происходит активное звездообразование \cite{park2012far}.

\begin{figure}[ht]
\center{\includegraphics[width=0.6\linewidth]{gb_serpens.jpg}}
\hfill
\caption{Туманность в созвездии Змея, изображение от космического телескопа Гершель (Hershel)}
\label{fig:area}
\end{figure}


Расстояние до туманности оценивается по-разному. Та её часть, которая относится к созвездию Орёл, расположена на расстоянии 225$\pm$55 парсек от Земли. Область, относящаяся к Змее, несколько дальше. Согласно измерениям параллакса, проведённым на радиоинтерферометре VLBA, она находится на расстоянии 415$\pm$25 парсек \cite{park2012far}.

Несмотря на то, что исследуемая область известна наличием звездообразования, ни одна звезда в ней не идентифицирована как относящаяся к типу Т Тельца. Это связано с расположением туманности близко к галактической плоскости и недостатком наблюдений в нужных спектральных диапазонах.

Мы рассматривали область неба, для которой прямое восхождение лежит в интервале от 17.96 до 18.72, а наклонение от -5 до 5.5. Вторая часть туманности не рассматривалась из-за отсутствия необходимых наблюдений.

\section{Метод поиска}
T Tauri звёзды имеют несколько характерных спектральных особенностей, а именно избыток излучения в инфракрасном, ультрафиолетовом и рентгеновском диапазоне. Рассматриваются ИК, оптический и УФ диапазоны, но отбор осуществляется по фотометрическим данным, не требуя наличия спектров.

Алгоритм поиска аналогичен алгоритму, использовавшемуся в работе \cite{AIGdC2014galex} для молекулярного облака Тельца.

Исследование проводится на основе ультрафиолетовых фотометрических данных космического телескопа GALEX. По ним, а также по фотометриям UCAC4 в оптическом диапазоне и 2MASS в ИК диапазоне, строятся двухцветные диаграммы. Положения объектов на диаграммах сравниваются с положениями известных и подтверждённых звёзд типа Т Тельца для получения критериев отбора кандидатов.

Далее первичный список очищается от лишних источников заведомо иного типа, проводится анализ результатов и финального списка.

\section{Актуальность}
Сам по себе список кандидатов в T Tauri звёзды нужен, чтобы определить объекты для наблюдения при дальнейшем поиске этих звёзд. Наблюдение этих кандидатов может стать одной из задач космического телескопа ВСО Спектр-УФ после его запуска. Оснащённый трёмя спектрографами, он будет способен проводить спектральные измерения слабых объектов, вплоть до 17 звёздной величины \cite{malkov2011scientific}, что позволит подтвердить или опровергнуть принадлежность кандидатов к типу T Tauri звёзд.

Область молекулярных облаков в созвездии Змея пока слабо изучена в ультрафиолетовом диапазоне. Несмотря на то, что она известна как область звездообразования, в ней обнаружено совсем мало звезд типа Т Тельца, и ни одна из них не находится в изучаемой части неба.



\chapter{T Tauri}

\section{Звёзды типа Т Тельца}
Звёзды типа Т Тельца - это маломассивные молодые звёзды, находящиеся на пути к главной последовательности. Обычно они находятся недалеко от отражательных или тёмных туманностей, оставшихся от газопылевого облака, из которого эти звёзды сформировались. Эти звёзды находятся в той части диаграммы Герцшпрунга - Рассела, которая соответствует звёздам с массами около 2-3 солнечных. С точки зрения звёздной эволюции они находятся в стадии гравитационного сжатия и, как молодые объекты, имеют близкую к солнечной металличность. (цитировать Додина) Характерными чертами также является избыточная эмиссия в ИК и УФ дапазонах.

Выделяют два подтипа этих звёзд: классические звёзды типа Т Тельца (classical T Tauri stars, CTTS) и звёзды типа Т Тельца со слабыми линиями (weak-lined T Tauri stars). Звёзды обоих подтипов находятся на одной стадии эволюции, имеют малую массу, и их металличность близка к солнечной. Различие состоит в том, что в спектрах классических звёзд типа Т Тельца присутствуют сильные эмиссионные линии, указывающие на то, что эти звёзды проявляют определённого рода активность. У звёзд типа Т Тельца со слабыми линиями эмиссионные линии гораздо слабее. Граница между подтипами проводится по эквивалентной ширине линии H$\alpha$. 

\section{Спектральные особенности}
Звёзды типа Т Тельца относятся к классу переменных звёзд. Первоначально они были выделены в отдельный класс на основе чисто спектроскопических характеристик: наличия эмиссии в линиях H$\alpha$ и Fe I, а также класс светимости IV-V. 

Сейчас выделяются следующие критерии принадлежности к типу:
\begin{itemize}
	\item Наличие поблизости тёмной или отражательной туманности;
	\item	Спектральный класс F5-M, класс светимости IV-V;
	\item	Эмиссия в линиях H и He I, а также нейтральных и однократно ионизированных металлов;
	\item	Сильная линия поглощения Li I 6707 A;
\end{itemize}

Присутствие линии Li указывает на молодость звёзд, так как согласно теоретическим расчётам литий быстро выгорает.

Эмиссионный спектр CTTS напоминает спектр солнечной хромосферы. Поэтому изначально считалось, что для них характерна высокая хромосферная активность. Но ожидаемое в этой модели сильное рентгеновское излучение не нашло экспериментального подтверждения.

В настоящее время считается, что спектральные особенности CTTS обусловлены наличием аккреционного диска. Если также предположить наличие магнитного поля, направление которого не совпадает с осью вращения звезды, то удаётся объяснить асимметричность эмиссионных линий. Предполагается, что они образуются на границе магнитосферы. Также аккреция вещества на звезду вызывает возникновение джетов. Это биполярные узконаправленные струи газа, истекающие со звезды. Они наблюдаются обычно в запрещённых линиях [SiI], [OI].

Наличие протопланетных дисков и магнитного поля у многих звёзд типа Т Тельца подтверждается наблюдениями.

Чтобы отличить звёзды типа Т Тельца от других, нам нужно знать характеристики их спектров, в особенности те из них, которые можно наблюдать в фотометриях. Как следствие существования аккреционного диска и истечения вещества на звезду, у TTS наблюдается избыток излучения в различных спектральных диапазонах, а именно:
\begin{itemize}
	\item Избыток в инфракрасном диапазоне вплоть до миллиметровых длин волн, обусловленный как собственным излучением нагретого диска, так и переизлучением поглощённого им излучения звезды и джетов.
	\item Избыток в оптическом диапазоне – свечение плазмы, нагретой до температуры 7000-10000 K. Это так называемое вуалирование – континуальное излучение нефотосферной природы. 
	\item Избыток в ультрафиолетовом диапазоне – свечение плазмы с температурой электронов от 10000 K до 50000 K, причём присутствует как излучение в континууме, так и различные эмиссионные линии: нейтральные атомы (H I, O I, C I), однократно (C II, Si II, Fe II, Mg II, O I) и многократно (C IV, N V, O VI) ионизованные атомы, молекулярный водород.
	\item Избыток в рентгеновском диапазоне, вызванный высокой активностью магнитосферы звезды, свечением короны и ударными волнами в аккреционном диске.
\end{itemize}
\section{Методы поиска}
 Изначально главными критериями поиска были лишь самые основные характеристики звёзд типа Т Тельца, как-то: близость к молекулярным облакам, избыток в инфракрасном диапазоне, присутствие магнитного поля. Позже, с появлением широкомасштабных обзоров неба, стали учитываться эквивалентная ширина линии H $\alpha$, собственные движения звёзд и избыток излучения в рентгеновском диапазоне. Также могут быть использованы оптические и инфракрасные данные и расрпределение спектральной энергии (SED) в этих диапазонах, в которых можно выделить черты, характерные для аккреционных дисков. Однако этими методами труднее обнаружить WTTS. Единственным действительно надёжным критерием является присутствие линии Li, как показатель молодости звезды.

Несмотря на то, что у звёзд типа Т Тельца присутствует существенный ультрафиолетовый избыток, большинство исследований, направленных на их поиск, проводилось в инфракрасном и рентгеновском диапазонах. С появлением ультрафиолетового обзора неба от миссии GALEX есть возможность использования и 





\chapter{Данные}
Данные мы брали из интернета, но это ничего страшного, они же в открытом дотупе лежат, и так и надо, зря что ли люди делали.

\section{GALEX}
Космическая миссия. О телескопе, звёздных величинах, фильтрах. Как выглядит спектр Т Таури в этих фильтрах.

О точности и покрытии неба. Картинка насчёт того как он покрывает мне область.

Первоначальный список источников из квадрата. Сколько. Какие данные

\section{2MASS и UCAC4}
Зачем нам вообще нужны эти данные. 2MASS для построения цветовых диаграмм, UCAC4 для оценки эффективных температур.

О каталогах, фильтрах, точности. В GALEX должно быть много ложных источников из-за их слабости и детектирования на пределе возможностей телескопа. Они пропадут при матчинге с нормальными каталогами
 
Кросс-идентификация первого списка с каталогами 2MASS и UCAC4. Радиус поиска. Какие откидывать? Новый список.

\section{Используемые инструменты}
casjobs для получения данных с галекса. visier для кросс-матчинга.


\chapter{Отбор}
Основной этап отбора кандидатов

\section{Эталонная выборка}
Что за звёзды, откуда, зачем.

\section{Цветовые диаграммы и критерии отбора}
Построение этих диаграмм, анализ. Нанесённые эталоны.

Критерии для трёх типов диаграмм. Почему именно эти диаграммы. Выбираем источники, для которых хотя бы одно из условий выполняется. Зачем нам третий критерий. И сослаться на Ану Инес.

\section{Результат и адекватность критериев}
Получили столько-то источников. Самый сильный критерий оказался третий (вроде)

Не упускают ли критерии известные звёзды типа Т Тельца? В нашей области проверить не на чем, но вот люди (Ана Инес) проверили на TMC и работает.


\chapter{Улучшение списка}
Теперь мы очистим список от источников, который могли попасть туда случайно. Это могут быть галактики, например, или горячие звёзды, потому что у них тоже присутствует избыток ультрафиолета.

\section{Удаление источников известного типа}
В первую очередь нужно очистить список от источников, которые заведомо не являются звёздами типа Т Тельца. Это могут быть галактики или звёзды, отношение которых к какому-то иному типу уже установлено.

С помощью сервиса Simbad можно узнать, чем является интересующий нас источник, имея лишь его координаты. Simbad (Set of Identifications, Measurements, and Bibliography for Astronomical Data ) -- это база данных об астрономических объектах, лежащих вне Солнечной системы. Таким образом, загрузив в неё список координат возможных кандидатов, мы можем узнать их тип, если он когда-либо и кем-либо был определён.

Из 302 источников нашего списка Simbad идентифицировал лишь 71, причём 56 из них являются галактиками, и 15 -- звёздами. Галактики должны быть выброшены из списка. Все звёзды не отнесены ни к какому типу, поэтому их следует оставить.  

Остальные источники не идентифицированы вовсе, то есть могут являться любым объектом. Они тоже, разумеется, должны быть оставлены в списке.

\section{Поиск галактик по собственным движениям}
Многие неидентифицированные Simbad источники могут также быть галактиками. Чтобы избавиться от них, мы можем сравнить их собственные движения. Те объекты, чьи собственные движения очень малы, то есть лежат в пределах погрешности наблюдений, с высокой вероятности лежат вне нашей Галактики.

Чтобы определить собственные движения, нужно заснять интересующую нас область и сравнить положения объектов с положениями, зафиксированными более ранними наблюдениями. Это может быть каталог DSS, который составлен на основе наблюдений, проведённых более 50 лет назад.

К сожалению, нам не удалось найти свежих фотографий неба в нужной области. 

В каталоге UCAC4 содержится информация о собственных движениях, но в него входят только объекты, заведомо являющиеся звёздами, поэтому он не может нам помооочь.

\section{Оценка эффективных температур}
Некоторые попавшие в список источники могут иметь ультрафиолетовый избыток просто потому что у них очень большая температура. Оцениваем её в VOSA, отбрасываем все источники горячее 7000 К.



\chapter{Анализ}


\section{Диаграммы цвет-интенсивность}

Если построить диаграмму цвет-интенсивность \ref{fig:color-magnitude}, то можно увидеть, насколько отличается блеск кандидатов от блеска остальных источников в области и от эталонной выборки. Видно, что звёзды выборки гораздо ярче как кандидатов, так и всех звёзд области. 

Большинство кандидатов очень слабые, их звёздные величины находятся на грани чувствительности GALEX. Также видно, что цвет FUV-NUV у кандидатов может быть отрицательным, и также просто отличаться от эталонных. Это даёт представление о количестве кандидатов, пришедших в выборку с третьим, неультрафиолетовым критерием.

Источников на рисунке \ref{fig:color-magnitude} меньше, чем в итоговом списке, так как многие как кандидаты, так и остальные звёзды не имеют FUV измерений.

\begin{figure}[ht]
\center{\includegraphics[width=0.6\linewidth]{colormagnitude.png}}
\hfill
\caption{Диаграмма цвет-интенсивность. Красным отмечены звёзды эталонной выборки, синим -- отобранные кандидаты}
\label{fig:color-magnitude}
\end{figure}

\section{Распределение спектральной энергии}

Сервис VOSA, использовавшийся для оценки эффективных температур, может строить распределение спектральной энергии, SED. Мы можем построить такую характеристику для эталонных звёзд, например, для T Tau, которая также входит в этот список, и сравнить с ней SED кандидатов. 

Как видно из рисунка, распределения не всегда похожи. Однако у большинства кандидатов в ультрафиолетовом диапазоне присутствует пик, подобный пику у T Tau.

\begin{figure}[ht]
\begin{minipage}[ht]{0.49\linewidth}
\center{\includegraphics[width=1\linewidth]{T_Tau.png} \\ }
\end{minipage}
\hfill
\begin{minipage}[ht]{0.49\linewidth}
\center{\includegraphics[width=1\linewidth]{obj144.png} \\ }
\end{minipage}
\begin{minipage}[ht]{0.49\linewidth}
\center{\includegraphics[width=1\linewidth]{obj146.png} \\ }
\end{minipage}
\hfill
\begin{minipage}[ht]{0.49\linewidth}
\center{\includegraphics[width=1\linewidth]{obj126.png} \\ }
\end{minipage}
\caption{Распределение спектральной энергии (SED) для T Tau и некоторых из кандидатов}
\label{fig:sed}
\end{figure}

\section{Оценка поглощения}

Межзвёздное поглощение в исследуемой области определяется в основном наличием молекулярных облаков. Однако, наблюдения миссии GALEX относятся только к периферии этих облаков, где поглощение значительно ниже. По некоторым оценкам, его величина $A_v < 0.5$ \cite{AIGdC2014galex} по другим $A_v < 0.5$ \cite{park2012far}, и таким поглощением можно пренебречь.

\section{Расположение}
Картинки с координатами и собственными движениями.

\section{Классические и со слабыми линиями}
В соответствии с результатами, полученными в аналогичном исследовании \cite{AIGdC2014galex}, WTTS и CTTS также находятся в разных областях двухцветной диаграммы FUV-NUV vs J-K:
\begin{itemize}
	\item Положение WTTS определяется линией $FUV - NUV = -(3.88 \pm 0.61)(J - K) + (5.64 \pm 0.55)$, среднеквадратичное отклонение равно 0.59.
	\item CTTS имеют нормальное распределение по оси $J - K$ со средним значением $a = 1.4$ и дисперсией $\sigma = 0.4$. По оси $FUV - NUV$ они распределены равномерно.
\end{itemize}

\begin{figure}[ht]
\center{\includegraphics[width=0.6\linewidth]{trend.png}}
\hfill
\caption{Двухцветная диаграмма FUV - NUV vs J - K. Красным обозначены звёзды эталонной выборки, чёрным -- кандидаты. Синими прямыми выделена область, в которой расположены WTTS.}
\label{fig:trend}
\end{figure}

Теперь можно распределить кандидаты в группы согласно этим результатам. К звёздам типа Т Тельца со слабыми линиями относим те, что лежат между штриховыми прямыми на рисунке \ref{fig:trend} (границы взяты втрое шире, чем коридор ошибок приближения), и имеют FUV~-~NUV > 3.5. К классическим T Tauri отнесём источники, у которых 0 < FUV~-~NUV < 3.5. Остальные кандидаты, включая не имеющие FUV, будем считать кандидатами в T Tauri звёзды без более детальной классификации.





\chapter{Выводы}
Итак, мы получили финальный список кандидатов в звёзды типа Т Тельца, всего их целых N штук.

Мы сделали всё, что мы сделали, мы молодцы. 

Теперь можно понаблюдать все кандидаты со Спектра-УФ, чтобы выяснить, действительно ли они типа T Tauri. Но можно и улучшить список ещё, если очень хочется, а именно -- пронаблюдать эти источники на БТА и уточнить их звёздные величины в видимом диапазоне.

Спасибо мама, папе, Ленке, Вове, Лохматому и Булке за моральную поддержку и веру в меня, спасибо спасибо.






\addcontentsline{toc}{chapter}{Список литературы}
\bibliography{Bib}     %% имя библиографической базы (bib-файла) 

\end{document}
